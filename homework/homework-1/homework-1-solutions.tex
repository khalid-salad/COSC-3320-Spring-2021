\documentclass{article}
\usepackage{homework}

\begin{document}
\begin{titlepage}\pagenumbering{gobble}
    \begin{center}
        {\scshape\LARGE University of Houston\par}
        \vspace{1cm}
        {\scshape\Large Homework 1 \par}
        \vspace{1.5cm}
        {\huge\bfseries COSC 3320 \par}
        {\huge\bfseries Algorithms and Data Structures \par}
        \vspace{0.5cm}
        {\large\bfseries Gopal Pandurangan\par}
        \vspace{2cm}
        {\Large Solutions\par}
    \end{center}
\end{titlepage}
\vspace*{\fill}\begin{center}{\Huge This page intentionally left blank.}\end{center}\vspace*{\fill}\thispagestyle{empty}\clearpage
\pagenumbering{arabic}

\section{Class Questions}
\subsection{January 26}

\begin{question}
    Show that any degree $d$ polynomial $p(n) = a_0n^d + a_1n^{d - 1} + \dots + a_{n-1}n + a_n$, with $a_0 > 0$, is $\bigO{n^d}$. (5 points)
\end{question}

\begin{solution}
    \begin{align*}\lim_{n\to\infty}\frac{a_0n^d + a_1n^{d - 1} + \dots + a_{n-1}n + a_n}{n^d}
         & =\lim_{n\to\infty}a_0 + \frac{a_1}{n} + \dots + \frac{a_{n-1}}{n^{d-1}} + \frac{a_n}{n^d} \\
         & = a_0
    \end{align*}
\end{solution}

\begin{question}
    Show that, for any $a > 1$, $n^b = \bigO{a^n}$. (5 points)
\end{question}

\begin{solution}
    First, this is clearly true if $b < 0$. Otherwise, we can apply the limit definition. Say $b$ is an integer. Then
    \begin{align*}\lim_{n\to\infty}\frac{n^b}{a^n}
         & = \lim_{n\to\infty}\frac{bn^{b-1}}{a^n\ln{a}}        \\
         & = \lim_{n\to\infty}\frac{b(b-1)n^{b-2}}{a^n\ln^2{a}} \\
         & \phantom{..}\vdots                                   \\
         & = \lim_{n\to\infty}\frac{b!}{a^n\ln^b{a}}            \\
         & = 0
    \end{align*}
    If, on the other hand, $b$ is not an integer, then $b = \floor{b} + r$, and
    \begin{align*}\lim_{n\to\infty}\frac{n^b}{a^n}
         & = \lim_{n\to\infty}\frac{bn^{b-1}}{a^n\ln{a}}                     \\
         & = \lim_{n\to\infty}\frac{b(b-1)n^{b-2}}{a^n\ln^2{a}}              \\
         & \phantom{..}\vdots                                                \\
         & = \lim_{n\to\infty}\frac{b(b-1)\dots(1+r)(r)b^{r-1}}{a^n\ln^b{a}} \\
         & = 0
    \end{align*}

    Alternatively, we can approach this more elegantly by considering the series \[S_n = \sum_{n=0}^{\infty}\frac{n^b}{a^n}\] If this series converges, then \[\lim_{n\to\infty}\frac{n^b}{a^n} = 0\] Apply the ratio test on $\sfrac{n^b}{a^n}$
    \begin{align*}\lim_{n\to\infty}\frac{\sfrac{(n + 1)^b}{a^{n + 1}}}{\sfrac{n^b}{a^n}}
         & = \lim_{n\to\infty}\frac{a^n(n+1)^b}{a^{n+1}n^b}           \\
         & = \lim_{n\to\infty}\frac{1}{a}\left(1+\frac{1}{n}\right)^b \\
         & = \frac{1}{a}
    \end{align*}
    The above limit is less than 1, since $a > 1$, hence the series converges by the ratio test. Thus, \[\lim_{n\to\infty}\frac{n^b}{a^n} = 0\] and $n^b=\bigO{a^n}$.
\end{solution}

\begin{question}
    Show that $a^{\log_b{n}} = n^{\log_b{a}}$.
\end{question}

\begin{solution}
    Let $x = \log_b{a}\log_b{n}$. Then
    \begin{align*}b^x
         & = b^{\log_b{a}\log_b{n}}                 \\
         & = \left(b^{\log_b{a}}\right)^{\log_b{n}} \\
         & = a^{\log_b{n}}
    \end{align*}
    Similarly,
    \begin{align*}b^x
         & = b^{\log_b{a}\log_b{n}}                 \\
         & = b^{\log_b{n}\log_b{a}}                 \\
         & = \left(b^{\log_b{n}}\right)^{\log_b{a}} \\
         & = n^{\log_b{a}}\qedhere
    \end{align*}
\end{solution}

\subsection{January 28}
\begin{question}
    Prove by mathematical induction that the gcd algorithm given in class is correct. You may assume $\gcd(a,b)=\gcd(b, a \bmod b)$. (5 points)
\end{question}

\begin{solution}
    Carefully formulate the induction hypothesis: we wish to show that, for all $n$, the algorithm given in class (which we will refer to as the Euclidean Algorithm going forward) for determining $\gcd(a, b)$ for any $a\leq n$, $b\leq n$, is correct. Notice that this is equivalent to the statement ``the Euclidean algorithm correctly determines $\gcd(a, b)$ for any non-negative integers $a$ and $b$''. From here, the proof is straightforward: the base case is trivial. Assume this is true for all $k < n$. Now, consider $\gcd(a, b)$ for $a\leq n$ and $b\leq n$. Without loss of generality, assume $a \geq b$ (since we can simply swap $a$ and $b$ if this is untrue).

    If $a < n$, we are done by our induction hypothesis. Otherwise, $a = n$. Now, if $b = n$, we are again done (since then $\gcd(a, b) = n$). Otherwise, the Euclidean algorithm updates $\gcd(a, b) \gets \gcd(b, a \bmod b)$. Since $b < n$, correctness follows from the induction hypothesis.
\end{solution}

\subsection{February 4}
\begin{question}
    Solve $T(n) = 3T(\sfrac{n}{2}) + n^2$. (5 points)
\end{question}

\begin{solution}
    Write $a = 3$, $b=2$, and $f(n) = n^2$. Then $af(n/b) = 3(\sfrac{n}{2})^2=\sfrac{3n^2}{4}$, hence $c =\sfrac{3}{4} < 1$. Thus, $T(n) = \bigTh{f(n)} = \bigTh{n^2}$.
\end{solution}

\begin{question}
    Show the correctness of \texttt{MergeSort} using Induction. You may assume the \texttt{Merge} subroutine is correct. (5 points)
\end{question}

\begin{solution}
    Let $n$ be the size of the array to be sorted. The induction is on $n$.
    The base case (i.e., $n \leq 1$)  is trivial.

    Induction hypothesis: Assume that the algorithm is correct on arrays of size less than $n$.
    We show that it is correct for an array of size $n$.
    Consider \texttt{MergeSort} on an array of length $n$. \texttt{MergeSort} is then recursively called on subarrays of length $\sfrac{n}{2}$, which are then sorted by our induction hypothesis. Then, by the correctness of the \texttt{Merge} procedure, the input array of length $n$ must be sorted.
\end{solution}

\begin{question}
    Argue that QuickSort cannot take more than $\binom{n}{2} = \sfrac{n(n - 1)}{2}$ comparisons. (5 points)
\end{question}

\begin{solution}
    In QuickSort, in every recursive call, one of the comparison elements has to be a pivot.
    Once the recursive call is finished, the pivot is never again compared (as it is not part of the parititioned
    elements). Hence no two elements are compared more than once. Since there are only
    $\binom{n}{2} = \sfrac{n(n - 1)}{2}$ elements, that is the worst-case number of comparisons.
\end{solution}

\section{Textbook Exercises}
\begin{exercise}{4.1}
    (10 points)
    Prove the asymptotic bound for the following recurrences by using
    induction. Assume that base cases of all the recurrences are
    constants i.e., $ T(n) = \bigTh{1}$, for $n < c$ where $c$ is some
    constant.
    \begin{enumerate}[label=(\alph*)]
        \item $T(n) \leq 2 T(n/2) + n^2$. Then, $T(n) = \bigO{n^2 \log n}$.
    \end{enumerate}
\end{exercise}

\begin{solution}
    The base case is given. Suppose $T(k) \leq ck^2\log{k}$ for all $k < n$. Then
    \begin{align*}T(n)
         & \leq 2T(n/2) + n^2                                                                        \\
         & \leq 2c\left(\frac{n}{2}\right)^2\log\frac{n}{2} + n^2\hbox{ by the induction hypothesis} \\
         & = \frac{c}{2}n^2\left(\log{n} - \log{2}\right) + n^2                                      \\
         & = \frac{c}{2}n^2\left(\log{n} - 1\right) + n^2                                            \\
         & = \frac{c}{2}n^2\log{n} -\frac{c}{2}n^2 + n^2                                             \\
         & = \frac{c}{2}n^2\log{n} + \left(1 - \frac{c}{2}\right)n^2                                 \\
         & \leq \frac{c}{2}n^2\log{n}\hbox{ for $c \geq 2$}                                          \\
         & \leq cn^2\log{n}\qedhere
    \end{align*}
\end{solution}

\begin{exercise}{4.3(6)}
    (5 points)
    Solve the following recurrences. Give the answer in terms of \emph{Big-Theta} notation.  Solve up to constant
    factors, i.e., your answer must give the correct function for $T(n)$, up to constant factors.
    You can  assume constant base cases, i.e., $T(1) = T(0) = c$, where $c$ is a positive constant.
    You can ignore floors and ceilings. You can use the DC Recurrence Theorem if it applies.
    \begin{enumerate}[start=6, label=\arabic*.]
        \item $T(n) = 4 T(n/2) + n^3$
    \end{enumerate}
\end{exercise}

\begin{solution}
    Write $a = 4$, $b=2$, and $f(n) = n^3$. Then $af(n/b) = 4(\sfrac{n}{2})^3=\sfrac{n^2}{2}$, hence $c =\sfrac{1}{2} < 1$. Thus, $T(n) = \bigTh{f(n)} = \bigTh{n^3}$.
\end{solution}

\begin{exercise}{4.4}
    (25 points)
    You are given an array consisting of $n$ numbers. A popular element is an element
    that occurs (strictly) \textbf{more than} $n/2$ times in the array. Give an algorithm that finds
    the popular element in the array if it exists, otherwise it should output ``NO''. Your algorithm should take no more than $2n$ comparisons. (As usual, we only count the comparisons between array elements.)
    Give pseudocode, argue its correctness, and show that your algorithm indeed takes no more than $2n$ comparisons.
    (Hint: Use a decrease and conquer strategy, similar to the celebrity problem.)
\end{exercise}

\begin{solution}
    Let $A$ denote our array and define $\Call{\nameref*{alg:boyermoore}}{A}$, so that, if a popular element of $A$ exists, it returns it, as follows
    \begin{algorithm}[H]
        \caption[\AlgName{Popular-Candidate}]{Determine if a popular element candidate exists}
        \label{alg:boyermoore}
        \begin{algorithmic}[1]
            \Function{\nameref*{alg:boyermoore}}{$A$}
            \If{$|A| = 1$}
            \State \Return $A[0]$
            \ElsIf{$|A| = 2}$
                \If{$A[0] = A[1]$}
                \State \Return $A[0]$
                \Else
                \State \Return \Null
                \EndIf
                \Else
                \If{two distinct elements exist in $A$}
                \State remove them
                \Else
                \State \Return $A[0]$
            \EndIf
            \EndIf
            \EndFunction
        \end{algorithmic}
    \end{algorithm}
    The key intuition here is to observe that, if $p$ is a popular element, then it is still popular when two \emph{distinct} elements of $A$ are removed. The proof is straightforward: if neither removed element is $p$, there are now $n - 2$ elements remaining, but $\sfrac{n}{2} > \sfrac{(n - 2)}{2}$ occurrences of $p$. If exactly one element is $p$, then there are $\sfrac{(n - 1)}{2} > \sfrac{(n - 2)}{2}$ occurrences of $p$, and it is still popular. Since the elements are distinct, they cannot \emph{both} be $p$. Correctness follows directly from this: algorithm~\nameref*{alg:boyermoore} will return the popular element if it exists. If it does not exist, we must determine so, as in the following pseudocode

    \begin{algorithm}[H]
        \caption[\AlgName{Popular-Element}]{Return a popular element, if it exists, else return \Null}
        \label{alg:checkboyermoore}
        \begin{algorithmic}[1]
            \Function{\nameref*{alg:checkboyermoore}}{$A$}
            \State $n\gets\card{A}$
            \State $\texttt{candidate} \gets \Call{\nameref{alg:boyermoore}}{A}$
            \State $\texttt{count} \gets 0$
            \ForEach{$\texttt{ele} \in A$}
            \If{$\texttt{ele} = \texttt{candidate}$}
            \State $\texttt{count} \gets \texttt{count} + 1$
            \EndIf
            \EndForEach
            \If{$\texttt{count} > \sfrac{n}{2}$}
            \State \Return \texttt{candidate}
            \Else
            \State \Return \Null
            \EndIf
            \EndFunction
        \end{algorithmic}
    \end{algorithm}

    A note: \nameref*{alg:boyermoore} can  be easily implemented to take only $n$ comparisons. Keep
    one element as a ``candidate'' element  (can be empty) throughout the algorithm. Initially this is set  to ``empty.''
    We also maintain another integer variable called ``count'' which counts the number of occurrences
    of the candidate element.  count is initially set to 0. We then scan the  array (starting from the first element), one
    element at a time. If the candidate element is set to null, we set it to the currently scanned element.
    Otherwise, we compare the candidate element with the current element and if they are equal, we increment
    the count by 1; if they are not equal, we decrement the count by 1. If the count becomes zero, then
    we also reset the candidate element to null.  Note that this algorithm takes at most $n$ comparisons in total.

    Checking the candidate further as in  Algorithm \nameref*{alg:checkboyermoore} takes additional $n$ comparisons.
    So the total number of comparisons is at most $2n$.

    (15 points for getting the main ideas of the algorithm and algorithm explanation; 5 points for pseudocode
    and 5 points for analysis).


    %This can be done in a variety of ways (the implementation details of which will vary from language to language), but the basic idea is to have one index $i$ at the first element and a second index $j$ at the second. Increment $j$ until the elements differ, then remove the elements.
\end{solution}

\begin{exercise}{5.1}
    Determine the total number of comparisons that each of the following algorithms takes on $S = [8, 2, 6, 7, 5, 1, 4, 3]$.
    \begin{itemize}[nosep]
        \item \AlgName{SimpleSort} (5 points)
        \item \AlgName{MergeSort} (5 points)
        \item \AlgName{QuickSort}  (5 points)
    \end{itemize}
    Show the steps of the algorithm when calculating the number of comparisons.
\end{exercise}

\begin{solution}
    (5 points each; just 1 point if only the final correct answer is given; 4 points for justification --- don't need
    to show all steps).

    \AlgName{SimpleSort}
    \begin{enumerate}[nosep]
        \item $[{\color{red}8},{\color{red}2},6,7,5,1,4,3]\rightarrow [{\color{red}2},{\color{red}8},6,7,5,1,4,3]$
        \item $[{\color{red}2},8,{\color{red}6},7,5,1,4,3]$
        \item $[{\color{red}2},8,6,{\color{red}7},5,1,4,3]$
        \item $[{\color{red}2},8,6,7,{\color{red}5},1,4,3]$
        \item $[{\color{red}2},8,6,7,5,{\color{red}1},4,3]\rightarrow [{\color{red}1},8,6,7,5,{\color{red}2},4,3]$
        \item $[{\color{red}1},8,6,7,5,2,{\color{red}4},3]$
        \item $[{\color{red}1},8,6,7,5,2,4,{\color{red}3}]$
        \item $[1,{\color{red}8},{\color{red}6},7,5,2,4,3]\rightarrow [1,{\color{red}6},{\color{red}8},7,5,2,4,3]$
        \item $[1,{\color{red}6},8,{\color{red}7},5,2,4,3]$
        \item $[1,{\color{red}6},8,7,{\color{red}5},2,4,3]\rightarrow [1,{\color{red}5},8,7,{\color{red}6},2,4,3]$
        \item $[1,{\color{red}5},8,7,6,{\color{red}2},4,3]\rightarrow [1,{\color{red}2},8,7,6,{\color{red}5},4,3]$
        \item $[1,{\color{red}2},8,7,6,5,{\color{red}4},3]$
        \item $[1,{\color{red}2},8,7,6,5,4,{\color{red}3}]$
        \item $[1,2,{\color{red}8},{\color{red}7},6,5,4,3]\rightarrow [1,2,{\color{red}7},{\color{red}8},6,5,4,3]$
        \item $[1,2,{\color{red}7},8,{\color{red}6},5,4,3]\rightarrow [1,2,{\color{red}6},8,{\color{red}7},5,4,3]$
        \item $[1,2,{\color{red}6},8,7,{\color{red}5},4,3]\rightarrow [1,2,{\color{red}5},8,7,{\color{red}6},4,3]$
        \item $[1,2,{\color{red}5},8,7,6,{\color{red}4},3]\rightarrow [1,2,{\color{red}4},8,7,6,{\color{red}5},3]$
        \item $[1,2,{\color{red}4},8,7,6,5,{\color{red}3}]\rightarrow [1,2,{\color{red}3},8,7,6,5,{\color{red}4}]$
        \item $[1,2,3,{\color{red}8},{\color{red}7},6,5,4]\rightarrow [1,2,3,{\color{red}7},{\color{red}8},6,5,4]$
        \item $[1,2,3,{\color{red}7},8,{\color{red}6},5,4]\rightarrow [1,2,3,{\color{red}6},8,{\color{red}7},5,4]$
        \item $[1,2,3,{\color{red}6},8,7,{\color{red}5},4]\rightarrow [1,2,3,{\color{red}5},8,7,{\color{red}6},4]$
        \item $[1,2,3,{\color{red}5},8,7,6,{\color{red}4}]\rightarrow [1,2,3,{\color{red}4},8,7,6,{\color{red}5}]$
        \item $[1,2,3,4,{\color{red}8},{\color{red}7},6,5]\rightarrow [1,2,3,4,{\color{red}7},{\color{red}8},6,5]$
        \item $[1,2,3,4,{\color{red}7},8,{\color{red}6},5]\rightarrow [1,2,3,4,{\color{red}6},8,{\color{red}7},5]$
        \item $[1,2,3,4,{\color{red}6},8,7,{\color{red}5}]\rightarrow [1,2,3,4,{\color{red}5},8,7,{\color{red}6}]$
        \item $[1,2,3,4,5,{\color{red}8},{\color{red}7},6]\rightarrow [1,2,3,4,5,{\color{red}7},{\color{red}8},6]$
        \item $[1,2,3,4,5,{\color{red}7},8,{\color{red}6}]\rightarrow [1,2,3,4,5,{\color{red}6},8,{\color{red}7}]$
        \item $[1,2,3,4,5,6,{\color{red}8},{\color{red}7}]\rightarrow [1,2,3,4,5,6,{\color{red}7},{\color{red}8}]$
    \end{enumerate}

    \AlgName{MergeSort}
    \begin{enumerate}[nosep]
        \item $[{\color{red}8}] [{\color{red}2}]\rightarrow[2, 8]$
        \item $[{\color{red}6}] [{\color{red}7}]\rightarrow[6, 7]$
        \item $[{\color{red}2},8] [{\color{red}6},7]\rightarrow[2]$
        \item $[2,{\color{red}8}] [{\color{red}6},7]\rightarrow[2, 6]$
        \item $[2,{\color{red}8}] [6,{\color{red}7}]\rightarrow[2, 6, 7, 8]$
        \item $[{\color{red}5}] [{\color{red}1}]\rightarrow[1, 5]$
        \item $[{\color{red}4}] [{\color{red}3}]\rightarrow[3, 4]$
        \item $[{\color{red}1},5] [{\color{red}3},4]\rightarrow[1]$
        \item $[1,{\color{red}5}] [{\color{red}3},4]\rightarrow[1, 3]$
        \item $[1,{\color{red}5}] [3,{\color{red}4}]\rightarrow[1, 3, 4, 5]$
        \item $[{\color{red}2},6,7,8] [{\color{red}1},3,4,5]\rightarrow[1]$
        \item $[{\color{red}2},6,7,8] [1,{\color{red}3},4,5]\rightarrow[1, 2]$
        \item $[2,{\color{red}6},7,8] [1,{\color{red}3},4,5]\rightarrow[1, 2, 3]$
        \item $[2,{\color{red}6},7,8] [1,3,{\color{red}4},5]\rightarrow[1, 2, 3, 4]$
        \item $[2,{\color{red}6},7,8] [1,3,4,{\color{red}5}]\rightarrow[1, 2, 3, 4, 5, 6, 7, 8]$
    \end{enumerate}

    \AlgName{QuickSort}
    \begin{enumerate}[nosep]
        \item $[{\color{red}8},{\color{red}2},6,7,5,1,4,3]\rightarrow [2], [{\color{red}8}], []$
        \item $[{\color{red}8},2,{\color{red}6},7,5,1,4,3]\rightarrow [2,6], [{\color{red}8}], []$
        \item $[{\color{red}8},2,6,{\color{red}7},5,1,4,3]\rightarrow [2,6,7], [{\color{red}8}], []$
        \item $[{\color{red}8},2,6,7,{\color{red}5},1,4,3]\rightarrow [2,6,7,5], [{\color{red}8}], []$
        \item $[{\color{red}8},2,6,7,5,{\color{red}1},4,3]\rightarrow [2,6,7,5,1], [{\color{red}8}], []$
        \item $[{\color{red}8},2,6,7,5,1,{\color{red}4},3]\rightarrow [2,6,7,5,1,4], [{\color{red}8}], []$
        \item $[{\color{red}8},2,6,7,5,1,4,{\color{red}3}]\rightarrow [2,6,7,5,1,4,3], [{\color{red}8}], []$
        \item $[{\color{red}2},{\color{red}6},7,5,1,4,3]\rightarrow [], [{\color{red}2}], [6]$
        \item $[{\color{red}2},6,{\color{red}7},5,1,4,3]\rightarrow [], [{\color{red}2}], [6,7]$
        \item $[{\color{red}2},6,7,{\color{red}5},1,4,3]\rightarrow [], [{\color{red}2}], [6,7,5]$
        \item $[{\color{red}2},6,7,5,{\color{red}1},4,3]\rightarrow [1], [{\color{red}2}], [6,7,5]$
        \item $[{\color{red}2},6,7,5,1,{\color{red}4},3]\rightarrow [1], [{\color{red}2}], [6,7,5,4]$
        \item $[{\color{red}2},6,7,5,1,4,{\color{red}3}]\rightarrow [1], [{\color{red}2}], [6,7,5,4,3]$
        \item $[{\color{red}6},{\color{red}7},5,4,3]\rightarrow [],[{\color{red}6}],[7]$
        \item $[{\color{red}6},7,{\color{red}5},4,3]\rightarrow [5],[{\color{red}6}],[7]$
        \item $[{\color{red}6},7,5,{\color{red}4},3]\rightarrow [5,4],[{\color{red}6}],[7]$
        \item $[{\color{red}6},7,5,4,{\color{red}3}]\rightarrow [5,4,3],[{\color{red}6}],[7]$
        \item $[{\color{red}5},{\color{red}4},3]\rightarrow[4][{\color{red}5}][]$
        \item $[{\color{red}5},4,{\color{red}3}]\rightarrow[4,3][{\color{red}5}][]$
        \item $[{\color{red}4},{\color{red}3}]\rightarrow[3][{\color{red}4}][]$
    \end{enumerate}
\end{solution}

\begin{exercise}{5.9}
    (25 points)
    Given three SORTED (in ascending order) arrays $A[1..n], B[1..n]$,
    and $C[1..n]$, each containing $n$ numbers, give an $\bigO{\log n}$-time algorithm (again, counting the number of comparisons) to find the $\nth{n}$ smallest number of all $3n$
    elements in arrays $A, B$, and $C$.
\end{exercise}

\begin{solution}
    Let $i = \floor{\sfrac{n}{3}}$ and define $a=A[i]$, $b = B[i]$, $c = C[i]$. Without loss of generality, say $a = \min(a, b, c)$. Now, there are at most $3\floor{\sfrac{n}{3}} < n$ elements in $A$, $B$, and $C$ less than $a$, hence we can remove these $\floor{\sfrac{n}{3}}$ smallest elements from $A$.

    \[A = [\underbrace{a_0, a_1, \dots, a_{i-1}}_{\text{remove these}}, a_i, \dots, a_{n - 1}]\]

    In the same vein, set $i' = 2\floor{\sfrac{n}{3}}$ and let $b' = B[i']$. Since $b$ is sorted and $a < b$, $a < b < b'$, hence there are at least $\floor{\sfrac{n}{3}} + 2\floor{\sfrac{n}{3}}$ elements less than $b'$, and we can remove all elements to the right of $b'$

    \[B = [b_0, b_1, \dots, b_{i'-1}, \underbrace{b_{i'}, \dots, b_{n - 1}}_{\text{remove these}}]\]

    And similarly for $C$

    \[C = [c_0, c_1, \dots, c_{i'-1}, \underbrace{c_{i'}, \dots, c_{n - 1}}_{\text{remove these}}]\]

    Leaving us with 3 arrays
    \begin{align*}
        A' & = [a_i, a_{i + 1}, \dots, a_{n - 1}] \\
        B' & = [b_0, b_1, \dots, b_{i'-1}]        \\
        C' & = [c_0, c_1, \dots, c_{i'-1}]
    \end{align*}
    each of length $2\floor{\sfrac{n}{3}}$. This removal requires a constant number of comparisons, and we will perform this $\bigO{\log{n}}$ times (since we remove approximately a third of the elements at each step), for a total $\bigO{\log{n}}$ comparisons.

    (10 points for the main ideas of the algorithm, in particular, showing how one can reduce the size
    of the arrays by a constant factor and reducing it to the subproblems of the same kind. 10 points
    for details; 5 points for analysis of $O(\log n)$ time.)


\end{solution}

\end{document}

